\documentclass[a4paper,twocolumn]{article}

\usepackage[english]{babel}
\usepackage[utf8]{inputenc}
\usepackage{graphicx}
\usepackage{caption}
\usepackage{fullpage}
\usepackage{titlesec}

\titleformat*{\section}{\large\bfseries}
\titleformat*{\subsection}{\normalsize\bfseries}
\titleformat*{\subsubsection}{\small\bfseries}

\newcommand{\D}{\mathcal{D}}
\newcommand{\T}{\mathcal{T}}
\newcommand{\X}{\mathcal{X}}
\newcommand{\Y}{\mathcal{Y}}

\title{A Survey on Transfer Learning $-$ summary}
\author{Matěj Nikl}

\begin{document}
\maketitle
As the title suggests, this paper focuses on transfer learning.
A few definitions:

\begin{itemize}
    \item a \textbf{Domain} $\D = \{\X, P(X)\}$
        \begin{itemize}
            \item a feature space $\X$
            \item a marginal probability distribution $P(X)$, where $X = \{x_1, \ldots, x_n\} \in \X$
            \item $\D_S$, $\D_T$ $-$ source and target domains
        \end{itemize}
    \item a \textbf{Task} $\T = \{\Y, f(\cdot)\}$ (given a domain $\D$)
        \begin{itemize}
            \item a label space $\Y$
            \item an objective predictive function $f(\cdot)$, which is not observed but can be learned from the training data
            \item $\T_S$, $\T_T$ $-$ source and target tasks
        \end{itemize}
    \item \textbf{Training Data} consist of pairs $\{x_i, y_i\}$
        \begin{itemize}
            \item $x_i \in \X$ and $y_i \in \Y$
            \item $D_S$, $D_T$ $-$ source and target data with $n_S$ and $n_T$ instances each
            \item in most cases $0 \le n_T \ll n_S$
        \end{itemize}
    \item \textbf{Transfer Learning} $-$ given $\D_S$, $\T_S$, $\D_T$ and $\T_T$, transfer learning aims to help improve the learning of the target predictive function $f_T(\cdot)$ in $\D_T$ using the knowledge in $\D_S$ and $\T_S$, where $\D_S \ne \D_T$, or $\T_S \ne \T_T$
        \begin{itemize}
            \item when $\D_S = \D_T$ and $\T_S = \T_T$, the learning problem becomes a traditional machine learning problem
        \end{itemize}
\end{itemize}

In other words, transfer learning aims to extract a knowledge from one or more source tasks (and their domains) and applies the knowledge to a target task.

One should not confuse transfer learning with multi-task learning $-$ multi-task learning tries to learn all of the source and target tasks simultaneously (thus treating them all equally), transfer learning cares the most about the target task.

There are three main research issues:
\begin{enumerate}
    \item \textbf{What} to transfer $-$ which part of the knowledge can be transferred across domains or tasks $-$ some of it may be specific for individual domains or tasks, some may be common
    \item \textbf{How} to transfer $-$ learning algorithms need to be developed to transfer the knowledge
    \item \textbf{When} to transfer $-$ in which situations should transferring be done as well as in which it should not be done $-$ when source and target domains are not related to each other, transferring the knowledge may even hurt the performace $-$ a phenomenon called \textit{negative transfer}
\end{enumerate}

Based on the definition of transfer learning, its relationship can be summarized into three sub-settings:

\begin{enumerate}
    \item \textbf{Inductive Transfer Learning}
        \begin{itemize}
            \item $\T_S \ne \T_T$
            \item a few labeled data in $\D_T$ are required to \textit{induce} $f_T(\cdot)$ for use in $\D_T$
        \end{itemize}
        Further categorization:
        \begin{enumerate}
            \item A lot of labeled data in $\D_S$ $-$ similar to the multi-task learning
            \item No labeled data in $\D_S$ $-$ similar to the self-taught learning
        \end{enumerate}
    \item \textbf{Transductive Transfer Learning}
        \begin{itemize}
            \item $\D_S \ne \D_T$, $\T_S = \T_T$
            \item A lot of labeled data in $\D_S$, no labeled data in $\D_T$ (thus some unlabeled data in $\D_T$ must be available)
        \end{itemize}
        Further categorization:
        \begin{enumerate}
            \item $\X_S \ne \X_T$
            \item $\X_S = \X_T$, but $P(X_S) \ne P(X_T)$, is related to:
                \begin{itemize}
                    \item domain adaptation for knowledge transfer in text classification
                    \item sample selection bias
                    \item co-variate shift
                \end{itemize}
        \end{enumerate}
    \item \textbf{Unsupervised Transfer Learning}
        \begin{itemize}
            \item $\T_S \ne \T_T$, $\Y_S$ and $\Y_T$ are not observable
            \item different focus $-$ clustering, dimensionality reduction, density estimation
            \item the predicted labels are latent variables, such as clusters or reduced dimensions
        \end{itemize}
\end{enumerate}

Approaches to the above three different settings can be summarized into four cases based on what to transfer:

\begin{enumerate}
    \item \textbf{Instance-based transfer learning} $-$ assumes that certain parts of the data in $\D_S$ can be reused for learning in $\D_T$ by re-weighting or importance sampling
    \item \textbf{Feature-representation-transfer} $-$ learning a ``good'' feature representation for $\D_T$, thus (hopefully) improving the performance significantly
    \item \textbf{Parameter-transfer} $-$ assumes that $\T_S$ and $\T_T$ share some parameters or prior distributions of the hyper-parameters of the models
    \item \textbf{Relational-knowledge-transfer} $-$ assumes that relationship among the data in $\D_S$ and $\D_T$ are similar
\end{enumerate}

\section{Inductive Transfer Learning}
\subsection{Transferring Knowledge of Instances}
$\D_S$ data cannot be used directly, however certain parts possibly can $-$ together with a few labeled data in $\D_T$.

TrAdaBoost boosting algorithm (extension of AdaBoost) addresses this problem, assumes $\X_S = \X_T$ and $\Y_S = \Y_T$, but $P(X_S) \ne P(X_T)$. It iteratively re-weights the $\D_S$ data to reduce the effect of the ``bad'' source data while encourage the ``good'' source data to contribute more for $\D_T$.

There are also other algorithms, e.g. a heuristic method to remove ``misleading'' training examples from $\D_S$ based on the difference between conditional probabilities $P(y_S|x_S)$ and $P(y_T|x_T)$.

\subsection{Transferring Knowledge of Feature Representations}
There are different strategies for different types of $\D_S$ data for finding ``good'' feature reprezentations to minimize domain divergence and classification or regression model error.


\subsubsection{Supervised Feature Construction}
\subsubsection{Unsupervised Feature Construction}
\subsection{Transferring Knowledge of Parameters}
\subsection{Transferring Relational Knowledge}

\section{Transductive Transfer Learning}
\subsection{Transferring Knowledge of Instances}
\subsection{Transferring Knowledge of Feature Representations}

\section{Unsupervised Transfer Learning}
\subsection{Transferring Knowledge of Feature Representations}

\end{document}

\documentclass[a4paper,twocolumn]{article}

\usepackage[english]{babel}
\usepackage[utf8]{inputenc}
\usepackage{graphicx}
\usepackage{hyperref}
\usepackage{amsmath}
\usepackage{amssymb}
\usepackage{fullpage}
\usepackage{titlesec}

\titleformat*{\section}{\large\bfseries}
\titleformat*{\subsection}{\normalsize\bfseries}
\let\originalleft\left
\let\originalright\right
\renewcommand{\left}{\mathopen{}\mathclose\bgroup\originalleft}
\renewcommand{\right}{\aftergroup\egroup\originalright}

\title{Online Incremental Feature Learning with Denoising Autoencoders$-$ summary}
\author{Matěj Nikl}

\begin{document}
\maketitle
Determining the optimal model complexity, i.e. the number of features for DAE is a nontrivial problem. Moreover, in large-scale and online datasets the cross-validation method enabling the hyper-parameter search is even more challenging.

This adaptive feature learning algorithm tries to address the given issue by:
\begin{enumerate}
    \item adding and training new features to minimize the residual of the objective function
    \item merging similar features to avoid redundant feature representations
\end{enumerate}
Having a variable number of features gives a model the ability to make itself more/less complex according to the (even changing) dataset needs.

\section{Denoising Autoencoder}
The standard Denoising Autoencoder is used as a building block for incremental feature learning, it consists of four components:
\begin{itemize}
    \item a conditional distribution $q(\mathbf{\widetilde{x}|x})$, that corrupts $\mathbf{x}$ into $\mathbf{\widetilde{x}}$
    \item an encoding function $f(\mathbf{\widetilde{x}}) = \sigma(\mathbf{W\widetilde{x} + b}) \equiv \mathbf{h} \in \mathbb{R}^N$, which gives a representation of input data $\mathbf{x}$
    \item an decoding function $g(\mathbf{h}) = \sigma(\mathbf{W}^T\mathbf{h + c}) \equiv \mathbf{\widehat{x}} \in \mathbb{R}^D$, which recovers the data from the representation $\mathbf{h}$
    \item a differentiable (cross-entropy) cost function $\mathcal{L}(\mathbf{x}) = \mathcal{H}(\mathbf{x}, \mathbf{\widehat{x}}) \in \mathbb{R}$, which computes the dissimilarity between input and reconstruction
\end{itemize}

\section{Incremental Feature Learning}
The incremental feature learning algorithm needs to address the following problems:
\begin{enumerate}
    \item \textit{when} to add and merge features
    \item \textit{how} to add and merge features
\end{enumerate}
In order to deal with these problems, the authors have chosen the following procedure:
\begin{enumerate}
    \item form a set $B$ of hard training examples: $B \leftarrow B \cup \{\textbf{x}\}$ if $\mathcal{L}(\textbf{x}) > \mu$, which is used to greedily initialize the new features
    \item train and add new features when there are enough examples in $B$: $|B| > \tau$
\end{enumerate}
In this paper $\mu$ was set as average of the objective function for recent $10000$ examples, $\tau$ was also set as 10000.


\end{document}

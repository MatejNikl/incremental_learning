\documentclass[a4paper]{article}

\usepackage[english]{babel}
\usepackage[utf8]{inputenc}
\usepackage[super]{nth}
% \usepackage{graphicx}
\usepackage{fullpage}
% \pagenumbering{gobble}

\title{Incremental Learning $-$ GoodAI}
\author{Matěj Nikl}

\begin{document}
\maketitle

% \begin{abstract}
%     Incremental learning yolo
% \end{abstract}

\section{Project Description}
The project's main goal is to think of and implement a machine learning
algorithm with the incremental learning feature, i.e. a model trained on a \nth{1}
task should be able to use the gained knowledge to converge on a \nth{2} task faster
(without forgetting the \nth{1} task), compared to a model learning the \nth{2}
task from scratch.

It will be demonstrated on a set of different tasks in very similar domains
(small noisy images with shapes in them).

\section{Work in July}
To gain some insight of what has been done in this area and to have a foundation
to build on top off I have read, understood and summarized into a (usually)
2 page long summaries these research papers:

\begin{itemize}
    \item \textbf{Learning to Compose Neural Networks for Question Answering},
        2016 $-$ neural network modules being arranged into a tree-like
        structure for each question to answer it correctly
    \item \textbf{Neural Module Networks}, 2016 $-$ preliminary work of the
        previous paper
    \item \textbf{Progressive Neural Networks}, 2016 $-$ neural networks
        tackling the problem of transferring the knowledge without forgetting
        from previously learned tasks to a new one (incremental learning)
    \item \textbf{A Survey on Transfer Learning}, 2009
    \item \textbf{The Cascade-Correlation Learning Architecture}, 1990
        $-$ a dynamic architecture, creating a \textit{cascade} of neurons and
        greedily training them, without the need to specify the network's
        topology
    \item \textbf{Convolutional Neural Fabrics}, 2016 $-$ a complex 3D
        trellis-like convolutional structure, embedding an exponentially large
        number of CNN architectures (via massive parameter sharing), allowing
        for choosing the correct one (or even ensemble of them) by standard
        back-propagation algorithm
    \item \textbf{Distilling the Knowledge in a Neural Network}, 2015 $-$ a way
        of transferring the knowledge from a cumbersome model (or an ensemble of
        models) to a smaller model, while preserving the performance
    \item \textbf{Learning Without Forgetting}, 2016 $-$ an incremental learning
        of new tasks without forgetting the old ones, using the distillation and
        fine-tuning
    \item \textbf{Online Incremental Feature Learning with Denoising Autoencoders},
        2012 $-$ a shallow DAE dynamically adjusting its hidden layer's size
        to perform well on a (online) dataset
\end{itemize}

\section{Work in August}
The project's goal should be fulfilled in this month by implementing a
experimental version of a machine learning algorithm with the incremental
learning feature. A detailed explanation of the implemented algorithm is required
as well, so that anybody is able to understand it properly.

\end{document}

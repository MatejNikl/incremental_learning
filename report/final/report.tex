\documentclass[a4paper,twocolumn]{article}

\usepackage[english]{babel} \usepackage[utf8]{inputenc} \usepackage[super]{nth}
% \usepackage{graphicx}
\usepackage[cm]{fullpage}
% \pagenumbering{gobble}

\title{Incremental Learning $-$ Final Report} \author{Matěj Nikl}

\begin{document} \maketitle

% \begin{abstract}
% Incremental learning is a hard problem to solve.
% \end{abstract}

\section{Introduction}
The task can be put simply $-$ implement a experimental architecture capable of
incremental learning.

That means to be able to leverage knowledge gained from learning previous
task(s) to learn the current task faster, while not forgetting the previous
ones.

I chose to re-implement the model described in the paper \textit{Learning
without Forgetting}, because I liked the idea and it felt like a good way to go.

\section{Learning without Forgetting}
The basic idea of Learning without Forgetting (LwF) architecture is to divide
the neural network being used to solve the tasks into a \textit{shared} part,
which is shared across all tasks and \textit{specific} parts, which are
task-specific.

Each time a new task is being learned, the old specific parts are trained in
such a way that their responses to the new training data remain the same (even
though the shared part is changing), while the new specific part is trained to
solve the new task.

The whole training procedure can be summarized into these steps:
\begin{enumerate}
    \item Save all outputs of old specific parts with temperatured softmax as a
        activation function for the new training data
    \item Train just the new specific part to solve the new task (the shared
        part is frozen)
    \item Train the whole network:
        \begin{itemize}
            \item old specific parts are trained in such a way that their
                outputs remain the same (via KL Divergence loss) $-$ the
                \textit{soft} target
            \item new specific part is trained to solve the new task $-$ the
                \textit{hard} target
        \end{itemize}
\end{enumerate}

\section{Implementation details}
My implementation does exactly the same thing as was proposed in the mentioned
paper, results are, however, not as good.

\subsection{Multi-label output} One of the obstacles I have had to face is the
multi-label nature of the tasks. My solution for this problem is a output layer
with matching number of neurons with sigmoid activation function. Interpretation
is done by thresholding at 0.5.

This solution is (as I have been told) not optimal. One would have to have the
threshold for each class as a learnable parameter (i.e. the threshold 0.5 might
not be optimal for all cases).

Another solution could be doubling the number of classes $n$ $-$ having a neuron
firing for the case when class $C$ is present, as well as when class $C$ is not
present. Interpretation would then be done by selecting $n$ outputs with the
highest probability.

Even though the \textit{doubling} solution is fairly straight-forward to
implement, I have not done so, because this has nothing to do with ultimate root
of my worse results and I have spent my time struggling those.

\subsection{The Soft Target}


\end{document}

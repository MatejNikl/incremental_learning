\documentclass[a4paper,twocolumn]{article}

\usepackage[english]{babel} \usepackage[utf8]{inputenc} \usepackage[super]{nth}
% \usepackage{graphicx}
\usepackage[cm]{fullpage}
% \pagenumbering{gobble}

\title{Incremental Learning $-$ Final Report} \author{Matěj Nikl}

\begin{document} \maketitle

% \begin{abstract}
% Incremental learning is a hard problem to solve.
% \end{abstract}

\section{Introduction}
The task can be put simply $-$ implement a experimental architecture capable of
\textbf{incremental learning}.

What it means is to be able to leverage knowledge gained from learning previous
task(s) to learn the current task faster, while not forgetting the previous
ones.

I chose to re-implement the model described in the paper \textit{Learning
without Forgetting}, because I liked the idea and it felt like a good way to
try.

\section{Learning without Forgetting}
The basic idea of Learning without Forgetting (LwF) architecture is to divide
the neural network being used to solve the tasks into a \textit{shared} part,
which is shared across all tasks and \textit{specific} parts, which are
task-specific.

Each time a new task is being learned, the old specific parts are trained in
such a way that their responses to the new training data remain the same (even
though the shared part is changing), while the new specific part is trained to
solve the new task.

The whole training procedure can be summarized into these steps:
\begin{enumerate}
    \item Save all outputs of old specific parts with temperatured softmax as a
        activation function for the new training data
    \item Train just the new specific part to solve the new task (the shared
        part is frozen)
    \item Train the whole network:
        \begin{itemize}
            \item old specific parts are trained in such a way that their
                outputs remain the same (via KL Divergence loss) $-$ the
                \textit{soft} target
            \item new specific part is trained to solve the new task $-$ the
                \textit{hard} target
        \end{itemize}
\end{enumerate}

\section{Implementation details}
I tried to implement my version of the LwF
architecture as close as possible to the proposed version (in which I have
succeeded, I think), results do, however, differ a lot. I have then tried slight
modifications to the base architecture in hope of better results, but only with
limited or no success.

\subsection{Multi-label output}
One of the obstacles I have had to face is the multi-label nature of the tasks.
At first I used a solution which uses the sigmoid activation function as the
output, interpreted by thresholding at 0.5.

This solution is (as I have been told) not optimal. One would have to have the
threshold for each class as a learnable parameter (i.e. the threshold 0.5 might
not be optimal for all cases).

Next solution I have implemented is doubling the number of classes $n$ $-$
having a neuron firing for the case when class $C$ is present, as well as when
class $C$ is not present. Interpretation is then done by selecting $n$ outputs
with the highest value (regardless the output activation function).

\subsection{The Hard Target}
For sigmoid activation I have used averaged binary cross-entropy loss. For
softmax activation function (which can only by used with the \textit{doubling}
solution) I have used sum of multiclass cross-entropy for all present classes
(as done in the paper).

\subsection{The Soft Target}
The paper suggests to use temperatured softmax as the output activation
function together with the sum of KL Divergence losses for each old task. I have
also tried to use absolute difference as well as MSE as the loss function
together with sigmoid activation.

\subsection{Incremental learning loss}
The total loss is a weighted sum of the hard and the soft loss. This is a
extension to what is proposed in the paper, because they take just the sum of
those two. The intention behind the weights is to stimulate the ``not
forgetting'' by giving the soft loss higher weight.

\subsection{Convolutional shared part}
In order to solve the harder version of the tasks, some sort of CNN must be
used. Implementing it for the LwF architecture is no different from implementing
it on its own. One just has to swap the shared part of the network with a CNN,
everything else remains the same.

\section{Results}



\end{document}
